	\index{Id}
			The Kingdom of Id\index{Id!Kingdom of} is a large nationstate encompassing the majority of Northern Ullr, comprised of an expansive and complex hierarchy of provinces ruled by local lords, and assembled into a parliamentary monarchy. Its terrain is poor for farming though rich in mineral and natural wealth, as it is mountainous, heavily wooded, and cold. Some of the farthest northern Iddish Lords remain tribal chiefs, though most of Id has since modernized into a feudal, then imperial, and finally metropolitan society.
			
			\subsection{Iddish History}
			For much of its history, Id has seemed to be a monolithic entity, inscrutable to outside eye and violent beyond all restraint. The latter claims are exaggeration while the former is an outright fabrication. While Iddish history has been marked with violence due in part to the scarcity of their land and large systems of familial alliances, Id is in no way a singular entity beyond being coalesced under a single overarching government. From the beginning of their known history, Id has been fragmented, and it is only in the modern era that the nation has somewhat solidified. 
			
			\subsubsection{Early History}\index{Id!Early History}\par
			From when mankind first entered the Iddish realms, they were confronted with inhospitable terrain. The ground is rocky, and a large portion of the realm is densely forested, rugged, and generally difficult to grow crops in. Most importantly, the winters of Id are quite severe, tending towards blizzard and deep freeze. Early Iddish life would have been extremely difficult, and when the first cities were first flourishing the Iddiat\index{Id!Iddiat} were still hunting and gathering.\par
			
			While their lands were poor, the Iddiat became very proficient at archery, hunting, and survival. Taking refuge in caves, usually hot spring caves that Id is known for, the Early Iddiat managed to build their first tribal communities. These tribes were heavily patriarchal, and relied on dogsleds and archery in order to hunt game such as elk, moose, and occasionally bear. The lack of Iddish coast or river precluded fishing, but the Iddiat continued to live within the Northern Ullr Expanse\index{Northern Ullr Expanse}.\par 
			
			Early Id was marked with endemic warfare due to scarcity of game and fruit. Tribes would frequently raid in search of food, either other Iddiat tribes or any traders taking advantage of the central Iddish Expanse as an easier route between East and West Ullr. In Iddiat society, anyone who was not related to a tribe was considered a fair target for that tribe's raids.\par 
			
			Late in the Iddish Early History, these tribes began to mark out their territories as areas near where their wintering caves. Cave art depicting territorial maps have been found in since abandoned Iddiat Wintering Caves, as well as some other spiritual art, depicting animistic gods and mountain deities to the far north. Drawings of stellar constellations, such as the Ice Bear, or the Sky Tree, have also been found in their early art. Iddiat society began to settle down, and Ancient Skadunism took root within their society. To describe this, notable Iddish historian Alexei Burgensen has said "The Early Iddiat did not believe in their superstitions. They feared them."\par
			
			\subsubsection{Introduction of Agriculture}\par
			Proximity to Bialka lead to a number of imported concepts and customs from Bialkan culture, the first of which being agriculture. Though millet and sorghum are the crops native to Ullr, they grow poorly in the boreal Iddish realm, and thus never became a major crop. As the rest of the world proceeded into an Agricultural revolution, the Iddiat were stuck hunting and gathering. They populated modern Iddish lands only sparsely.\par 
			
			However, with the spread of potatoes, rye, and various other grains from Ma'at, the Iddiat were no longer restricted by a lack of food. Although Id has no major rivers, the hot springs in their caves were used to irrigate their new farmlands. Their number grew exponentially due to hardier crop, and the mountain complexes became all the more vital. However, the increase in population meant that even these new crops were not enough to satisfy the food requirements of the Iddiat population.\par 
			
			\subsubsection{Iddiat Clans and Raiding of Bialka}\par
			The increase in population meant that the Iddiat communities could grow larger and more powerful, but also that they needed more food in order to survive. Though permanent mountain complexes for farming began to form, most men of a clan would embark on long raiding expeditions on dogsled to retrieve more food for the winter. The targets of these raids were usually other Iddiat Clans, but groups on the southeastern fringe of Iddish territory would often conduct raids into Bialkan territory.\par 
			
			These raids brought back not only food but Bialkan customs. Rather than basic pictograms, the Iddiat began to write in the Bialkan form, and use the ore they excavated from their mountain complexes in order to forge iron instruments and weapons. Quickly, the most powerful Iddiat clans became the southeastern ones due to their technological advantage and greater food supply recovered from raids into fertile Bialkan territory.\par
			
			Most notable of these clans was the Beliskner clan. Reaching its apex in power in the year [GIVE ME A FUCKING YEAR], it was one of the first of the tribes to adopt the formal system. Khan Harald Beliskner proudly inscribed a monument to his greatness before the mountain complex of Khan-Beylik. He notes on his achievements the tribute of eleven other more minor clans, and his ability to summon a host numbering in the thousands. His monument is one of the best preserved examples of its kind. Generally speaking, these monuments proclaim the achievements and the abilities of an Iddiat patriarch, in addition to their formal inscription of key parts of the Law of Taboo.
			
			\subsubsection{The Iddish Crusades}\par
			
			\subsubsection{Iddiat Participation in Skadunist Crusades}\par
			
			\subsubsection{Creation of Hearths}\par
			
			\subsubsection{Demirbjorn Revolution and Iddish High Kings}\par
			
			\subsubsection{Conspiracy of Id}\par
			
			\subsubsection{First Dissonance}\par
			
			\subsubsection{Reformative Era}\par
			
			\subsubsection{Kupperstar and Panserna Dynasties}\par
			
			\subsubsection{Second Dissonance}\par
			
			\subsubsection{Empire of Great Id}\par
			
			\subsubsection{Third Dissonance}\par
			
			\subsubsection{First Parliamentary Era}\par
			
			\subsubsection{Fourth Dissonance and Second Parliamentary Era}\par
			
			\subsubsection{Millennial Id and the Millennium War}\par
			
			\subsubsection{Modern Id}\par
			
			\subsection{Ancient Iddiat Customs}\par
			\subsubsection{Iddiat Houses and Tribes}\index{Id!Iddiat}\index{Id!Noble Houses of}\par
			The Iddiat Houses and Tribes were the adoptive familial groups that were headed by a patriarchal 'Father', and whose decisionmaking were largely dominated by the advice of said Father's 'sons', both adopted and biological. These sons would usually be ranked by age, with the eldest sons being considered more powerful and more intelligent than the younger. However, sons with proven aptitude could be considered spiritually older than their biologically older counterparts.\par 
			
			When an Iddiat Tribe settles in and becomes the de facto owner of a cave system, it was generally accepted that they had become a House, and that they should be afforded the respect deserving of one. Iddiat Houses were held to higher standards, being allowed less leniency in violation of Taboo due to their greater political power. Fathers of Iddiat Houses were also to be given gifts of higher worth, and their sons being expected to be more honorable in their exploits. Women of Iddiat houses were also of a greater marital value than those of Iddiat Tribes.\par 
			
			The women of Iddiat society were given a submissive role. Their duties were held in place by the Law of Taboo, and were relegated to most of the same duties as men but without the ability to intervene in the decision-making process. Women were also forbidden from being warriors or knights, and were otherwise relegated to being marital stock.\par 
			
			The collapse of a House or a Tribe can be brought about by the elimination of the Father and majority of high ranking sons, or the breakage of Taboo on multiple levels. Rarely does the breakage of Taboo occur. Should a House or Tribe collapse, its constituent members will be considered orphaned, and therefore not protected by most Taboo law. However, many tribes adopted orphans out of pity, though some orphans chose to leave the Iddish Realm entirely.
			
			\subsubsection{Law of Taboo}\index{Id!Taboo, Law of}
			The Law of Taboo is a fundamental component of Iddiat and Modern Iddish society. Though some portions of Taboo were later codified, they were originally a set of unwritten rules and customs agreed upon by all Iddish Tribes and Houses, entirely informal in nature.\par 
			
			Taboo is formulated in a way that describes how the ideal Father, the ideal Son, and the ideal Brother should act. A wide variety of beliefs from prehistoric Iddiat society are believed to have sanctified in Taboo. Usually taught in an ad hoc format, Taboo is passed down from Father to Son, and frequently communicated between tribesmen. In early days, this meant that Taboo varied widely in teaching, though as trans-iddish communication became easier and the importance of Taboo grew, these variations shrank. This coincided with several All-Fathers using their power to codify the most important Laws of Taboo.\par
			
			No penalties were described in the Law of Taboo, beyond violators of the law would no longer be protected by it, and thus were free to be dealt with however other Iddiat saw fit. If a violation of Taboo was not punished, that violation would itself become a part of the Law. Large archives and records of major events in Iddish History were compiled so that Fathers would know what had and had not been permitted by Sanctified Violations, as well as dissertations on what constituted a violation.\par
			
			\subsubsection{Filial Retinue}\index{Id!Filial Retinue}
			Iddiat and later Iddish familial groups were dominated by males, either taking a Father role or a Son role. The Father, being the patriarch of the group, was expected the act righteously and honorably with Taboo in mind. As the Iddiat tribes grew into Houses, this burden became too great for a singular man to bear, and Fathers began entrusting their Sons with these tasks.\par
			
			Sons were expected to act righteously and honorably with Taboo in mind as well, effectively becoming administrators, generals, guardians, and advisors to their Fathers, while acting as Fathers themselves to their sons. Brothers were not permitted to command any of their brothers, no matter younger or elder, but Fathers were permitted to command Sons.\par
			
			A man with more able bodied sons was considered a more powerful Father than one who had fewer. Smaller tribes may have only had five or six sons, while the most powerful Fathers had sons numbering in the thousands. Most famously, Allfather Iorek Chelikhan Anderssen boasted regularly of having well over ten thousand direct sons himself, with over half being biological.\par
			
			\subsubsection{Marital Negotiation}\index{Id!Martial Negotiation}
			Entrance into a tribe would be done through marriage and ritual adoption. When a marriage was being negotiated, both of the Iddiat Patriarchs would come to an agreement of how the two tribes would merge. Usually, this could result in either of the two fathers gaining a gift while the other absorbs the son, and any of that son's subservients, into his family, or the two tribes integrating into one, with the two patriarchs becoming sworn brothers. These are the first traces of the infamous Iddiat Houses\index{Id!Iddiat Houses}.\par
			
			\subsubsection{Ancient Iddish Mythology}\index{Id!Ancient Mythology}
			\subsubsection{Iddiat Dogsleds}\index{Trade!Ullr!Great Northern Road}
			\subsection{Modernity}\index{Id!Kingdom of}
			\subsubsection{Iddish Houses}\index{Id!Iddish Houses}
			\subsection{Modern Iddish Society}\index{Id!Modern Society}
			\subsection{Iddish Culture}\index{Id!Culture}