\index{Albion}

	Albion is a nation consisting primarily of the island of Albia, of western Ullr.  The island takes its name from the state, as opposed to vice versa, "Albion" originally being a dynastic name of sorts, describing a duchy that originated from the mainland.  Although Albion was forced to migrate across the East Sea against their will some thousand years ago, this proved to be fortuitous, as the relative isolation and hospitable terrain preserved the nation.  Many of its contemporaries, lacking the natural defense the sea provided, fell to the sword of invaders.  Albion expanded into the vacuum their demise created, which facilitated the rise of one of the greatest empires of the Middle and Imperial eras.

	Albion has been best known as one of the two founding members of the East Sea Commonwealth, for its extensive new world colonial empire, and for the invention of the railway.
	
\subsection{Early History}
	
	Albion is the sole survivor of four sister kingdoms, the other three being Elbion, Sertion, and Marion.  Evidence for the existence of the other three states exists and is not in dispute, however the story of their origin has no known historical basis and is considered by most historians to be legendary rather than factual.
	
	\textit{``The king of the realm had four sons.  In order to preserve the peace between his sons upon his death, he ordered that his realm be split in four, with one kingdom given to each son.  The eldest was granted the largest amount of land, named the Kingdom of Marion, and the youngest was granted the smallest kingdom, the Kingdom of Albion.  For a time, there was peace.  However, despite its small size, Albion grew richer and more powerful than the other three brothers.  Eventually the three became tired of being overshadowed by Albion, led my Marion they attacked Albion out of jealousy.  Albion avoided destruction by crossing the water, and in doing so saved the Albian nation from destruction at the hands the other three did not foresee.''} -Albian Legend
	
\subsection{Government}
	
	For most of recorded history, Albion has been governed by a monarch.  Succession was initially hereditary, however an elective succession was instituted to solve a succession crisis circa 370 AC.  Although the system was imperfect, it did prove useful for expanding Albion's influence in neighboring Kingdoms, as the succession code permitted the ascension of foreign royalty to the Albian throne.  In this way, personal unions could be formed, with future monarchs continuing to rule both countries.  The most notable instance of this is the union between Albion and Wodenburg, which is commonly cited as the founding of the East Sea Commonwealth.  It is important to note that this is not a retrospective label- the first recorded use of the name is believed to date back to only a few years after the union was first created.
	
	Historically the Albian monarchy was quite strong, however the same was not necessarily true of the nobility.  Until the 7th century Albion was a nominally feudal society, but this system had  coexisted with a limited civil service since the 4th century, primarily for the collection of taxes as well as a rudimentary secret police force.  One significant and exclusive (At least until the 9th century) power the nobility held was the right to participate in the election of the Albian monarch.